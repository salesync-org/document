\chapter*{Tóm tắt}
\label{summary}
Tài liệu này được ra đời với mục đích hướng dẫn người đọc cách triển khai các kết quả biên dịch và release của đề tài Thực tập dự án tốt nghiệp "Hệ thống quản lý khách hàng cho doanh nghiệp nhỏ" lên các máy chủ, hoàn thiện quá trình xuất bản ứng dụng đến người dùng.

Đề tài được thực hiện theo kiến trúc microservice, do đó tất cả service hoàn toàn có thể được triển khai độc lập. Sau khi kết thúc quá trình release như được hướng dẫn ở file \texttt{SOURCE/ReleasenGuide.pdf} , ta cần tiến hành đưa những sản phẩm sau đây lên một hoặc một số server khác nhau.
Cụ thể, những sản phẩm cần được triển khai bao gồm:
\begin{enumerate}
    \item Server \texttt{KeyCloak} (Docker Container): Dùng cho việc lưu trữ dữ liệu xác thực, thông tin cá nhân, quyền hạn của người dùng, thông tin doanh nghiệp.
    \item Service \texttt{authenticaion} (Java Spring Boot): Dùng để tương tác với máy chủ KeyCloak.
    \item Service \texttt{type-service} (Java Spring Boot): Giúp quản lý, khởi tạo, xây dựng và cấu hình cấu trúc các đối tượng dữ liệu cho công ty.
    \item Service \texttt{record-service} (Java Spring Boot): Lưu trữ thông tin chi tiết các bản ghi của những đối tượng dữ liệu đã cấu hình.
    \item Service \texttt{notification-service} (Java Spring Boot): Tiếp nhận và trao đổi thông báo giữa các service đến phía người dùng.
    \item Service \texttt{service-registry} (Java Spring Boot): Cổng đăng kí các server khả dụng cho từng service.
    \item Server \texttt{RabbitMQ} (Docker Container): Dùng cho việc giao tiếp bất đồng bộ giữa các service trong hệ thống Microservice.
    \item Server \texttt{PostgreSQL} (Database): Các database PostgreSQL riêng cho các service \texttt{notification}, \texttt{type-ervice}, \texttt{record-service}.
    \item \textbf{Storage Service} để lưu các tệp tin, hình ảnh của ứng dụng.
    \item Service \texttt{api-gateway} (Java Spring Boot): Cổng giao tiếp giữa client và các service khả dụng.
    \item Ứng dụng \texttt{Frontend} (Vite, React, TypeScript, TailwindCSS): Ứng dụng giao tiếp với người dùng cuối.
\end{enumerate}

Tất cả nội dung và sản phẩm thực hiện đồ án kể trên đều được đi kèm trong file \texttt{SourceCode.zip}. Đối với những database, hệ thống sẽ sử dụng dịch vụ \textbf{AWS RDS} (Free Tier).
Đối với front-end, hệ thống sẽ sử dụng \textbf{Vercel}. Đối với tất cả những service còn lại, hệ thống sẽ sử dụng những máy chủ \textbf{AWS EC2} (Free Tier) hoặc \textbf{GCP Computer Engine} để triển khai những Docker image đã khởi tạo.
Mọi hướng dẫn được thực hiện trên hệ điều hành CentOS 7.0 -- hệ điều hành phổ biến cho các Server chạy nhân Linux.