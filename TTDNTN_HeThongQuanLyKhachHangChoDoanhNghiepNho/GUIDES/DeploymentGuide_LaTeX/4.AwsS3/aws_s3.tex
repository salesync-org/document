\chapter{Triển khai AWS S3 và API liên quan đến storage}
\label{Chapter4}
Trước khi triển khai front-end, ta cần xây dựng service \texttt{AWS S3} (hoặc tương tự) để lưu trữ các file ảnh của trang web và của người dùng.\\
Sau đó, ta xây dựng một AWS API Gateway để thực hiện lệnh upload ảnh đến \texttt{AWS S3}.

\section{Cây tài nguyên AWS S3}
Xây dựng và tổ chức thư mục tài nguyên của \texttt{AWS S3} như sau:
\begin{enumerate}
    \item salesync
    \begin{enumerate}
        \item \texttt{avatars}: Chứa toàn bộ avatar (sau khi đã processed) của người dùng.
        \item \texttt{copmanies}: Chứa toàn bộ tài nguyên hình ảnh của tất cả doanh nghiệp.
        \item \texttt{preproccessed\_avatars}: Nơi lưu trữ những avatars do người dùng upload, chưa được processed.
        \item \texttt{system}: Những hình ẩnh mà ứng dụng cần sử dụng            
    \end{enumerate}
\end{enumerate}

\textbf{\textit{Chú ý:}} Tài nguyên này nên được đặt ở chế độ \textbf{Public}.
\section{Tạo hàm Lambda để process avatar}
Mỗi khi người dùng upload avatar, ta cần kích hoạt hàm lambda để xử lý ảnh sang nhiều kích thước khác nhau, sau đó lưu vào thư mục \texttt{avatars}.
Chi tiết về cách thiết lập quyền cho hàm lambda có thể được xem qua ở \href{https://www.geeksforgeeks.org/serverless-image-processing-with-aws-lambda-and-s3/}{đây}.

\section{Xây dựng API Gateway cho Storage}
Dựa vào hướng dẫn tại \href{https://aws.amazon.com/blogs/compute/patterns-for-building-an-api-to-upload-files-to-amazon-s3/}{đây}. Ta xây dựng một API Gateway resource có 2 API dành riêng cho việc upload 
avatars vào thư mục \texttt{preproccessed\_avatars} cũng như vào thư mục \texttt{companies}.

\section{Kết quả}
Sau khi đã cài đặt thành công, ta có tổng cộng $3$ api, $1$ để đọc tài nguyên \texttt{AWS S3}, $2$ để ghi vào các thư mục con.




